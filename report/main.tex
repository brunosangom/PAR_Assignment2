\documentclass{article}
\usepackage{graphicx}
\usepackage{caption}
\usepackage{geometry}
\usepackage{placeins}
\usepackage{enumitem}
\usepackage{tcolorbox}
\usepackage{multirow}
\usepackage{float}

% Set page margins
\geometry{a4paper, margin=2cm}

% Set paragraph and spacing
\setlength{\parindent}{0em} % No indentation (annoying)
\setlength{\parskip}{0.5em} % Small space between paragraph

% Redefine the caption format to remove "Figure *:"
\captionsetup[figure]{labelformat=empty}

\begin{document}

\title{PAR - Assignment 2 - Report}
\author{\normalsize Bruno Sánchez \& Jean Dié}
\date{\small 5th November 2024}

\maketitle

\section{Introduction to the Problem}

This assignment involves designing a plan for a robot chef operating in a restaurant’s kitchen to perform meal preparation tasks. The robot’s objective is to receive orders, gather necessary ingredients, execute cooking processes, and serve finished dishes. Working within a divided kitchen layout, the robot must navigate through designated areas for storage, preparation, cooking, and dishwashing while adhering to specific operational constraints.

The planning solution is implemented in the Planning Domain Definition Language (PDDL), defining actions, predicates, and conditions necessary for the robot to achieve its goals efficiently. This report includes an analysis of the problem, an overview of the PDDL implementation, test cases, and results highlighting the robot’s ability to adapt to various scenarios, manage resources, and optimize task sequences. The following sections delve into the detailed problem analysis, solution architecture, and results of scenario-based testing to evaluate the robot’s performance and adaptability.

\section{Problem Analysis}

To implement the model in \textit{PDDL}, we first need to define the predicates and actions that will be used. The model involves a chef robot operating within a grid-based kitchen layout, where it must cook and serve different kinds of dishes. The predicates define the state of the world (in this case, the kitchen and its elements), such as valid rooms and their adjacencies, the location of tools, and the condition of ingredients. Actions describe the possible operations the robot can perform, including moving around the kitchen, picking up items (tools, ingredients and dishes), and using them to assemble the required dishes. Additionally, the model must account for the relative priority of the orders and the limited amounts of ingredients.

\vspace{1em}

\begin{figure}[H]
    \centering
    \includegraphics[width=0.55\textwidth]{assets/kitchen-layout.png}
    \caption{Kitchen layout}
    \label{fig:kitchen}
\end{figure}
\newpage
\subsection{Predicates}\label{sec:pred}

\begin{itemize}[label=--, itemsep=0.05em]
    \item \textbf{at(Movable, Room):} A \textit{Movable} entity is located in \textit{Room}.
    \item \textbf{holding(Item):} The robot is currently holding \textit{Item}.
    \item \textbf{hand-free():} The robot's hand is free.
    \item \textbf{adjacent(Room1, Room2):} \textit{Room1} and \textit{Room2} are adjacent.
    \item \textbf{ingredient-stored(Ingredient):} \textit{Ingredient} is stored in the storage area.
    \item \textbf{ingredient-available(Ingredient):} \textit{Ingredient} is available somewhere in the kitchen.
    \item \textbf{ingredient-prepared(Ingredient):} \textit{Ingredient} is prepared.
    \item \textbf{ingredient-cooked(Ingredient):} \textit{Ingredient} is cooked.
    \item \textbf{is-storage-room(Room):} \textit{Room} is designated as the storage room.
    \item \textbf{is-cooking-room(Room):} \textit{Room} is designated as the cooking room.
    \item \textbf{is-serving-room(Room):} \textit{Room} is designated as the serving room.
    \item \textbf{is-preparation-room(Room):} \textit{Room} is designated as the preparation room.
    \item \textbf{is-dishwashing-room(Room):} \textit{Room} is designated as the dishwashing room.
    \item \textbf{is-cutting-room(Room):} \textit{Room} is designated as the cutting room.
    \item \textbf{is-mixing-room(Room):} \textit{Room} is designated as the mixing room.
    \item \textbf{tool-use-room(Tool, Room):} \textit{Tool} is used in \textit{Room}.
    \item \textbf{tool-clean(Tool):} \textit{Tool} is clean.
    \item \textbf{ingredient-prep-room(Ingredient, Room):} \textit{Ingredient} must be prepared in \textit{Room}.
    \item \textbf{ingredient-used-in-dish(Ingredient, Dish):} \textit{Ingredient} is used in \textit{Dish}.
    \item \textbf{require-prepared(Dish, Ingredient):} \textit{Dish} requires \textit{Ingredient} to be prepared.
    \item \textbf{require-cooked(Dish, Ingredient):} \textit{Dish} requires \textit{Ingredient} to be cooked.
    \item \textbf{prioritize-dish(Dish1, Dish2):} \textit{Dish1} has a higher priority than \textit{Dish2}.
    \item \textbf{next-dish(Dish):} \textit{Dish} is currently being cooked.
    \item \textbf{dish-prepared(Dish):} \textit{Dish} is fully prepared.
    \item \textbf{dish-served(Dish):} \textit{Dish} has been served.
    
\end{itemize}

\subsection{Actions}\label{sec:act}

\begin{itemize}[label=--, itemsep=0.05em]
    \item \textbf{move(Robot, From, To):} \textit{Robot} moves from room \textit{From} to an adjacent room \textit{To}.
    \item \textbf{pick-up(Robot, Item, Room):} \textit{Robot} picks up \textit{Item} from \textit{Room}.
    \item \textbf{take-from-storage(Robot, Ingredient, Dish, Room):} \textit{Robot} retrieves \textit{Ingredient} from the storage room \textit{Room} for its use in \textit{Dish}.
    \item \textbf{refill-ingredient(Robot, Ingredient, Room):} \textit{Robot} refills \textit{Ingredient} in the storage \textit{Room}.
    \item \textbf{drop-tool(Robot, Tool, Room):} \textit{Robot} drops \textit{Tool} in \textit{Room}, its designated use room.
    \item \textbf{drop-ingredient(Robot, Ingredient, Room):} \textit{Robot} places \textit{Ingredient} in \textit{Room}, either the ingredient's preparation room or the general preparation room.
    \item \textbf{prepare-ingredient(Robot, Ingredient, Tool, Room):} \textit{Robot} prepares \textit{Ingredient} in \textit{Room} using \textit{Tool}.
    \item \textbf{cook-ingredient(Robot, Ingredient, Room):} \textit{Robot} cooks \textit{Ingredient} in the cooking \textit{Room}.
    \item \textbf{assemble-dish(Robot, Dish, Room):} \textit{Robot} assembles \textit{Dish} from prepared ingredients in the preparation \textit{Room}.
    \item \textbf{serve-dish(Robot, Room, Dish):} \textit{Robot} serves \textit{Dish} in the serving \textit{Room}.
    \item \textbf{clean-tool(Robot, Tool, Room):} \textit{Robot} cleans \textit{Tool} in the dishwashing \textit{Room}.
    \end{itemize}

\section{Project Structure and PDDL Implementation}

The project is mainly contained in the \texttt{code} directory, which contains the necessary files for implementing and testing the model. It includes a \texttt{domain.pddl} file, which establishes the model's conditions, including predicates and actions; and the \texttt{problem-*.pddl} files, which define the initial conditions and goal states for each test case, allowing us to evaluate the model under various scenarios.

Additionally, each problem has a corresponding \texttt{problem-*.plan} file, which contains the PDDL plan generated by the planner, and a \texttt{problem-*.txt} file, which logs the planner's output, including computational metrics. To efficiently extract and analyze these metrics, we developed a Python script, \texttt{run\_plan.py}, which executes the planner for a chosen problem and logs its output to provide insights into the computational performance of the model.

To facilitate the development and testing of our PDDL files, we set up our project using \textbf{Visual Studio Code} with the PDDL extension. We utilized the \textbf{BFWS planner} to execute and evaluate our PDDL models, allowing us to test various scenarios and configurations efficiently.

In this section, we will only explain the \texttt{domain.pddl} implementation of the model. In the next section, \textit{Test Cases and Results}, we will show the multiple problem configurations that we have tested and the results attained with each of them.

\subsection{Requirements}
The PDDL requirements needed in the domain are:
\begin{enumerate}
    \item \texttt{:strips}, which indicates that the domain adheres to the basic action representation of the STRIPS framework.
    \item \texttt{:typing}, which allows the use of types for predicates and objects in the domain.
    \item \texttt{:negative-preconditions}, which enables the use of the \texttt{not} operator in the preconditions of actions.
    \item \texttt{:disjunctive-preconditions}, which allows for the use of the \texttt{or} operator in the preconditions of actions.
    \item \texttt{:conditional-effects}, which permits the specification of effects that depend on the state of the world, with the \texttt{when} operator.
    \item \texttt{:universal-preconditions}, which allows for preconditions that must hold for all objects of a specific type, with the \texttt{forall} operator.
    \end{enumerate}
\begin{figure}[ht]
    \centering
    \includegraphics[width=0.90\textwidth]{assets/requirements.png}
    \caption{Code: Requirements}
    \label{fig:req}
\end{figure}

\subsection{Types}
The types are used to categorize objects within a domain, providing a way to define relationships and constraints between different entities.
\begin{enumerate}
    \item \texttt{room - object}, which defines a type for locations or rooms.
    \item \texttt{movable - object}, which defines a type for movable entities.
    \item \texttt{robot - movable}, which specifies that the robot is a type of movable object.
    \item \texttt{item - movable}, which defines a general type `item' for anything the robot might hold.
    \item \texttt{ingredient, tool, dish - item}, which specifies that `ingredient', `tool', and `dish' are subtypes of `item'.    
\end{enumerate}
\begin{figure}[H]
    \centering
    \includegraphics[width=0.90\textwidth]{assets/types.jpeg}
    \caption{Code: Types}
    \label{fig:types}
\end{figure}

\subsection{Predicates}
The predicates are essentially variables that can take either a \textit{true} or \textit{false} value for each collection of \textit{parameters} that they are given. In our domain, the predicates are those discussed in Section~\ref{sec:pred}.
\begin{figure}[H]
    \centering
    \includegraphics[width=0.90\textwidth]{assets/predicates.png}
    \caption{Code: Predicates}
    \label{fig:pred}
\end{figure}

\subsection{Actions}
The actions are the possible steps that can be taken throughout a plan. Each of them recieves a number of \textit{parameters}, which are the objects that intervene in the action. Then, for the action to be allowed to happen, all of its \textit{preconditions} have to be met and, in that case, the \textit{effects} of the action can be applied. In our domain, the actions are those discussed in Section~\ref{sec:act}.
\subsubsection{Move}
The \texttt{move} action allows the robot to move between adjacent rooms of the kitchen.
\begin{itemize}
    \item \underline{Parameters:}
    \begin{itemize}
        \item \texttt{Robot}: The robot that is performing the move action.
        \item \texttt{From}: The room from which the robot is moving.
        \item \texttt{To}: The room to which the robot is moving.
    \end{itemize}
    \item \underline{Preconditions:}
    \begin{itemize}
        \item \textit{At(Robot, From)}: The robot must be located in the room \texttt{From}.
        \item \textit{Adjacent(From, To)}: The rooms \texttt{From} and \texttt{To} must be adjacent to each other.
    \end{itemize}
    \item \underline{Effects:}
    \begin{itemize}
        \item \textit{At(Robot, To)}: The robot is now located in the room \texttt{To}.
        \item \textit{¬At(Robot, From)}: The robot is no longer in the room \texttt{From}.
    \end{itemize}
\end{itemize}
\begin{figure}[ht]
    \centering
    \includegraphics[width=0.80\textwidth]{assets/move.png}
    \caption{Code: Move Action}
    \label{fig:act:move}
\end{figure}

\subsubsection{Pick-up}
The \texttt{pick-up} action allows the robot to take items (tools, ingredients or dishes) from the kitchen.

\begin{itemize}
    \item \underline{Parameters:}
    \begin{itemize}
        \item \texttt{Robot}: The robot that is performing the pick-up action.
        \item \texttt{Item}: The item that the robot intends to pick up.
        \item \texttt{Room}: The room where the robot and the item are located.
    \end{itemize}
    \item \underline{Preconditions:}
    \begin{itemize}
        \item \textit{At(Robot, Room)}: The robot must be located in the room \texttt{Room}.
        \item \textit{At(Item, Room)}: The item must also be located in the same room \texttt{Room}.
        \item \textit{Hand-free}: The robot must not be holding any other item.
    \end{itemize}
    \item \underline{Effects:}
    \begin{itemize}
        \item \textit{Holding(Item)}: The robot is now holding the item \texttt{Item}.
        \item \textit{¬Hand-free}: The robot's hand is no longer free (it is now holding an item).
        \item \textit{¬At(Item, Room)}: The item is no longer located in the room \texttt{Room}.
    \end{itemize}
\end{itemize}
\begin{figure}[ht]
    \centering
    \includegraphics[width=0.90\textwidth]{assets/pick-up.png}
    \caption{Code: Pick-Up Action}
    \label{fig:act:pick-up}
\end{figure}

\subsubsection{Take-from-storage}
The \texttt{take-from-storage} action allows the robot to take an ingredient from the storage and make it available in the kitchen.
\begin{itemize}
    \item \underline{Parameters:}
    \begin{itemize}
        \item \texttt{Robot}: The robot that is performing the take-from-storage action.
        \item \texttt{Ingredient}: The ingredient that the robot intends to take from storage.
        \item \texttt{Dish}: The dish that requires the ingredient.
        \item \texttt{Room}: The room where the robot and the storage are located.
    \end{itemize}
    \item \underline{Preconditions:}
    \begin{itemize}
        \item \textit{At(Robot, Room)}: The robot must be located in the room \texttt{Room}.
        \item \textit{Is-storage-room(Room)}: The room must be designated as a storage room.
        \item \textit{Ingredient-stored(Ingredient)}: The ingredient must currently be stored in the storage room.
        \item \textit{Hand-free}: The robot must not be holding any other item.
        \item \textit{Next-dish(Dish)}: The dish must be the next one to be prepared.
        \item \textit{Ingredient-used-in-dish(Ingredient, Dish)}: The ingredient must be required for the specified dish.
    \end{itemize}
    \item \underline{Effects:}
    \begin{itemize}
        \item \textit{Holding(Ingredient)}: The robot is now holding the ingredient \texttt{Ingredient}.
        \item \textit{Ingredient-available(Ingredient)}: The ingredient is now available for use.
        \item \textit{¬Hand-free}: The robot's hand is no longer free (it is now holding an ingredient).
        \item \textit{¬Ingredient-stored(Ingredient)}: The ingredient is no longer in storage.
    \end{itemize}
\end{itemize}
\begin{figure}[H]
    \centering
    \includegraphics[width=0.90\textwidth]{assets/take-from-storage.png}
    \caption{Code: Take-from-storage Action}
    \label{fig:act:take}
\end{figure}

\subsubsection{Refill-ingredient}
The \texttt{refill-ingredient} action restocks the storage with a specific ingredient that is no longer available.
\begin{itemize}
    \item \underline{Parameters:}
    \begin{itemize}
        \item \texttt{Robot}: The robot that is performing the refill action.
        \item \texttt{Ingredient}: The ingredient that the robot is refilling in storage.
        \item \texttt{Room}: The room where the robot and storage are located.
    \end{itemize}
    \item \underline{Preconditions:}
    \begin{itemize}
        \item \textit{At(Robot, Room)}: The robot must be located in the room \texttt{Room}.
        \item \textit{Is-storage-room(Room)}: The room must be designated as a storage room.
        \item \textit{¬Ingredient-stored(Ingredient)}: The ingredient must not currently be in storage.
        \item \textit{¬Ingredient-available(Ingredient)}: The ingredient must not be available in the kitchen.
    \end{itemize}
    \item \underline{Effects:}
    \begin{itemize}
        \item \textit{Ingredient-stored(Ingredient)}: The ingredient is now stored in the storage room.
    \end{itemize}
\end{itemize}
\begin{figure}[ht]
    \centering
    \includegraphics[width=0.80\textwidth]{assets/refill-ingredient.png}
    \caption{Code: Refill-ingredient Action}
    \label{fig:act:refill}
\end{figure}

\subsubsection{Drop-tool}
The \texttt{drop-tool} action makes the robot drop a tool. A tool can only be droped at its use room.
\begin{itemize}
    \item \underline{Parameters:}
    \begin{itemize}
        \item \texttt{Robot}: The robot performing the drop action.
        \item \texttt{Tool}: The tool that the robot intends to drop.
        \item \texttt{Room}: The room where the tool should be dropped.
    \end{itemize}
    \item \underline{Preconditions:}
    \begin{itemize}
        \item \textit{At(Robot, Room)}: The robot must be located in the room \texttt{Room}.
        \item \textit{Holding(Tool)}: The robot must be holding the tool \texttt{Tool}.
        \item \textit{Tool-use-room(Tool, Room)}: The tool must be used in the specified room \texttt{Room}.
    \end{itemize}
    \item \underline{Effects:}
    \begin{itemize}
        \item \textit{¬Holding(Tool)}: The robot is no longer holding the tool.
        \item \textit{Hand-free}: The robot’s hand is now free.
        \item \textit{At(Tool, Room)}: The tool is now located in the specified room \texttt{Room}.
    \end{itemize}
    \end{itemize}
\begin{figure}[ht]
    \centering
    \includegraphics[width=0.80\textwidth]{assets/drop-tool.png}
    \caption{Code: Drop-tool Action}
    \label{fig:act:drop-tool}
\end{figure}

\subsubsection{Clean-tool}
The \texttt{clean-tool} action allows the robot to clean a used tool in the dishwashing room.
\begin{itemize}
    \item \underline{Parameters:}
    \begin{itemize}
        \item \texttt{Robot}: The robot responsible for cleaning the tool.
        \item \texttt{Tool}: The tool to be cleaned.
        \item \texttt{Room}: The dishwashing area where the tool is cleaned.
    \end{itemize}
    \item \underline{Preconditions:}
    \begin{itemize}
        \item \textit{At(Robot, Room)}: The robot must be located in \texttt{Room}.
        \item \textit{Is-dishwashing-room(Room)}: The room \texttt{Room} must be designated as a dishwashing area.
        \item \textit{Holding(Tool)}: The robot must be holding the tool.
        \item \textit{¬Tool-clean(Tool)}: The tool must not already be clean.
    \end{itemize}
    \item \underline{Effects:}
    \begin{itemize}
        \item \textit{Tool-clean(Tool)}: The tool is now marked as clean.
    \end{itemize}
\end{itemize}
    \begin{figure}[ht]
    \centering
    \includegraphics[width=0.75\textwidth]{assets/clean-tool.png}
    \caption{Code: Clean-tool Action}
    \label{fig:act:clean-tool}
\end{figure}

\subsubsection{Drop-ingredient}
The \texttt{drop-ingredient} action makes the robot drop an ingredient. An ingredient can only be droped at its designated preparation room or at the general preparation room.
\begin{itemize}
    \item \underline{Parameters:}
    \begin{itemize}
        \item \texttt{Robot}: The robot performing the drop action.
        \item \texttt{Ingredient}: The ingredient that the robot intends to drop.
        \item \texttt{Room}: The room where the ingredient will be dropped.
    \end{itemize}
    \item \underline{Preconditions:}
    \begin{itemize}
        \item \textit{At(Robot, Room)}: The robot must be located in the room \texttt{Room}.
        \item \textit{Holding(Ingredient)}: The robot must be holding the ingredient \texttt{Ingredient}.
        \item \textit{Ingredient-prep-room(Ingredient, Room) \textbf{or} Is-preparation-room(Room)}: The room must either be the specific preparation room for the ingredient or a designated preparation room.
    \end{itemize}
    \item \underline{Effects:}
    \begin{itemize}
        \item \textit{¬Holding(Ingredient)}: The robot is no longer holding the ingredient.
        \item \textit{Hand-free}: The robot’s hand is now free.
        \item \textit{At(Ingredient, Room)}: The ingredient is now located in the specified room \texttt{Room}.
    \end{itemize}
\end{itemize}
    \begin{figure}[ht]
    \centering
    \includegraphics[width=0.90\textwidth]{assets/drop-ingredient.png}
    \caption{Code: Drop-ingredient Action}
    \label{fig:act:drop-ing}
\end{figure}

\subsubsection{Prepare-ingredient}
With the \texttt{prepare-ingredient} action, the robot prepares (cuts or mixes) an ingredient, depending on the designated preparation it requires.
\begin{itemize}
    \item \underline{Parameters:}
    \begin{itemize}
        \item \texttt{Robot}: The robot performing the preparation.
        \item \texttt{Ingredient}: The ingredient to be prepared.
        \item \texttt{Tool}: The tool required to prepare the ingredient.
        \item \texttt{Room}: The room where the ingredient is being prepared.
    \end{itemize}
    \item \underline{Preconditions:}
    \begin{itemize}
        \item \textit{At(Robot, Room)}: The robot must be located in the room \texttt{Room}.
        \item \textit{Ingredient-prep-room(Ingredient, Room)}: The ingredient must be prepared in this specified room.
        \item \textit{At(Ingredient, Room)}: The ingredient must be present in the room \texttt{Room}.
        \item \textit{Holding(Tool)}: The robot must be holding the required tool.
        \item \textit{Tool-use-room(Tool, Room)}: The tool must be usable in the specified room.
        \item \textit{Tool-clean(Tool)}: The tool must be clean before starting the preparation.
    \end{itemize}
    \item \underline{Effects:}
    \begin{itemize}
        \item \textit{Ingredient-prepared(Ingredient)}: The ingredient is now prepared.
        \item \textit{¬Tool-clean(Tool)}: The tool is no longer clean after preparing the ingredient.
    \end{itemize}
\end{itemize}
    \begin{figure}[H]
    \centering
    \includegraphics[width=0.90\textwidth]{assets/prepare-ingredient.png}
    \caption{Code: Prepare-ingredient Action}
    \label{fig:act:prepare-ing}
\end{figure}

\subsubsection{Cook-ingredient}
The \texttt{cook-ingredient} is used by the robot to cook an ingredient in the cooking room when it is required.
\begin{itemize}
    \item \underline{Parameters:}
    \begin{itemize}
        \item \texttt{Robot}: The robot performing the cooking action.
        \item \texttt{Ingredient}: The ingredient to be cooked.
        \item \texttt{Room}: The room where the cooking takes place.
    \end{itemize}
    \item \underline{Preconditions:}
    \begin{itemize}
        \item \textit{Is-cooking-room(Room)}: The room \texttt{Room} must be designated as a cooking room.
        \item \textit{At(Robot, Room)}: The robot must be located in \texttt{Room}.
        \item \textit{Ingredient-prepared(Ingredient)}: The ingredient must be prepared before it can be cooked.
        \item \textit{Holding(Ingredient)}: The robot must be holding the ingredient to cook it.
    \end{itemize}
    \item \underline{Effects:}
    \begin{itemize}
        \item \textit{Ingredient-cooked(Ingredient)}: The ingredient is now cooked.
    \end{itemize}
\end{itemize}
    \begin{figure}[ht]
    \centering
    \includegraphics[width=0.90\textwidth]{assets/cook-ingredient.png}
    \caption{Code: Cook-ingredient Action}
    \label{fig:act:cook-ing}
\end{figure}

\subsubsection{Assemble-dish}
With the \texttt{assemble-dish} action, the robot is able to put together all of the prepared and/or cooked ingredients in order to produce a dish.
\begin{itemize}
    \item \underline{Parameters:}
    \begin{itemize}
        \item \texttt{Robot}: The robot responsible for assembling the dish.
        \item \texttt{Dish}: The dish being assembled.
        \item \texttt{Room}: The preparation room where the assembly takes place.
    \end{itemize}
    \item \underline{Preconditions:}
    \begin{itemize}
        \item \textit{At(Robot, Room)}: The robot must be located in \texttt{Room}.
        \item \textit{Is-preparation-room(Room)}: The room \texttt{Room} must be designated as the preparation area.
        \item \textit{Forall Ingredients}: Ensures all required ingredients meet the preparation requirements for the dish:
        \begin{itemize}
            \item If the dish requires a prepared ingredient, it must be \textit{Ingredient-prepared} and located in \texttt{Room}.
            \item If the dish requires a cooked ingredient, it must be \textit{Ingredient-cooked} and located in \texttt{Room}.
        \end{itemize}
    \end{itemize}
    \item \underline{Effects:}
    \begin{itemize}
        \item \textit{Dish-prepared(Dish)}: The dish is now prepared.
        \item \textit{At(Dish, Room)}: The dish is now located in \texttt{Room}.
        \item \textit{Forall Ingredients}: For each ingredient used in the dish:
        \begin{itemize}
            \item The ingredient is no longer available, prepared, or cooked.
            \item The ingredient is no longer located in \texttt{Room}.
        \end{itemize}
    \end{itemize}
\end{itemize}
    \begin{figure}[H]
    \centering
    \includegraphics[width=0.90\textwidth]{assets/assemble-dish.png}
    \caption{Code: Assemble-dish Action}
    \label{fig:act:assemble-dish}
\end{figure}

\subsubsection{Serve-dish}
With the \texttt{serve-dish} action, the robot can finally serve a dish after it has been prepared.
\begin{itemize}
    \item \underline{Parameters:}
    \begin{itemize}
        \item \texttt{Robot}: The robot responsible for serving the dish.
        \item \texttt{Room}: The serving area where the dish is served.
        \item \texttt{Dish}: The dish to be served.
    \end{itemize}
    \item \underline{Preconditions:}
    \begin{itemize}
        \item \textit{Is-serving-room(Room)}: The room \texttt{Room} must be designated as the serving area.
        \item \textit{At(Robot, Room)}: The robot must be located in \texttt{Room}.
        \item \textit{Holding(Dish)}: The robot must be holding the dish.
        \item \textit{Dish-prepared(Dish)}: The dish must be fully prepared.
        \item \textit{Next-dish(Dish)}: The dish must be the next one in the serving order.
        \item \textit{Forall Other Dishes}: If there are any prioritized dishes before \texttt{Dish}, they must be served already.
    \end{itemize}
    \item \underline{Effects:}
    \begin{itemize}
        \item \textit{Dish-served(Dish)}: The dish is now marked as served.
        \item \textit{¬Holding(Dish)}: The robot is no longer holding the dish.
        \item \textit{Hand-free}: The robot’s hand is now free.
        \item \textit{¬Next-dish(Dish)}: The dish is no longer the next one in the serving order.
        \item \textit{Forall Other Dishes}: If \texttt{Dish} was prioritized before any other dish, that dish now becomes the next in line to be served.
    \end{itemize}
\end{itemize}
    \begin{figure}[ht]
    \centering
    \includegraphics[width=0.70\textwidth]{assets/serve-dish.png}
    \caption{Code: Serve-dish Action}
    \label{fig:act:serve-dish}
\end{figure}

\newpage
\section{Test Cases and Results}

We conducted tests on a total of four cases, the first three with increasing complexity, and the fourth to test differences in priority. With each test case, we iteratively test the different functionalities of the model.

For each of the four cases, we explain the objective of that test case, display a diagram of the initial and goal states, and briefly analyze the solution achieved with the planner. All of the solutions can be found in the \texttt{problem-*.plan} files attached to this report.

\subsection{Case 1: Basic}

To begin, we tested the fundamental functionalities of the model, focusing on the robot's ability to move around the kitchen, pick up items, prepare ingredients, assemble a dish and finally serve it. This initial test case features only one dish (\textit{sushi}).

\begin{figure}[ht]
    \centering
    %\includegraphics[width=0.55\textwidth]{assets/problem-1.drawio.png}
    \caption{Case 1: Basic}
    \label{fig:initial-state}
\end{figure}
\FloatBarrier

We can see that the movement action functions correctly with the adjacencies of the rooms, as well as the pick-up and drop-off actions, and all of the actions needed to process ingredients and dishes.

\subsection{Case 2: Multiple Dishes}

Subsequently, we incorporate an additional complexity into the model: serving more than one dish (\textit{sushi} and now, \textit{ramen}). This setup is designed to rigorously assess the robot's order priority system, and ability to serve multiple dishes in one plan.

\begin{figure}[H]
    \centering
    %\includegraphics[width=0.55\textwidth]{assets/problem-2.drawio.png}
    \caption{Case 2: Multiple Dishes}
    \label{fig:initial-state-dishes}
\end{figure}
\FloatBarrier

We now see that the robot does not start preparing the second dish until it has served the first one, and it successfully delivers both orders.

\subsection{Case 3: Multiple Servings A}

For the next case, we examine another of the features of the model: the limit on ingredient quantities. In order to do this, we utilize a test case similar to the previous one, but in this case, we introduce more than one serving per dish. The goal of this test case is to study whether the limit of ingredients (and the \texttt{refill-ingredient} action) work as expected.

\begin{figure}[H]
    \centering
    %\includegraphics[width=0.55\textwidth]{assets/problem-3.drawio.png}
    \caption{Case 3: Multiple Servings A}
    \label{fig:initial-state-servingsA}
\end{figure}
\FloatBarrier

In this case, we prove that the ingredient quantities are being taken into account as they should, having to restock each of them when they have already been used in a previous dish.

\textbf{Note}: On earlier iterations of the model, we tried to implement the ingredient quantities with numeric functions (\texttt{:fluents} requirement of PDDL), but the available free online planners have trouble supporting them. Therefore, we decided to simply limit the quantity of any ingredient to 1 by default, and have the \texttt{refill-ingredient} action in order to generate more units if needed. We had considered enforcing quantities with typing (for instance, have a type \texttt{fish} and in then create \texttt{fish1}, \texttt{fish2}... in the problem initialization). However, this greatly limited the domain, since the model would then only consider as possible ingredients the ones defined as types, instead of being flexible and allowing for ingredient definition for each problem. \textbf{We consider the limitation to 1 unit per ingredient to be more acceptable than the limitation of the domain to a specific set of ingredients.}

\subsection{Case 4: Multiple Servings B}

For the final case, we evaluate essentially the same scenario as in Case 3, but with a slight difference: a change in the priority of the dishes. This configuration allows us to confirm that the ordering system is functioning as expected, and also lets us test the relative efficiency of two very similar plans.

\begin{figure}[ht]
    \centering
    %\includegraphics[width=0.55\textwidth]{assets/problem-4.drawio.png}
    \caption{Case 4: Multiple Servings B}
    \label{fig:initial-state-servingsB}
\end{figure}
\FloatBarrier

It is now shown that the ordering system is indeed being correctly considered, and that the robot might act more efficiently with different priority settings.

\section{Performance Analysis}

The following table summarizes the results of the test cases, comparing the performance metrics of the model across different problems.

\begin{table}[ht]
    \centering
    \small
    \begin{tabular}{|c|c|c|c|c|}
        \hline
        \textbf{Problem} & \textbf{Plan Cost} & \textbf{Nodes Generated} & \textbf{Nodes Expanded} & \textbf{Total Time (ms)} \\
        \hline
        Problem-1 & 57 & 843 & 295 & 5 \\
                                   
        \hline
        Problem-2 & 102 & 2058 & 669 & 13 \\
                                   
        \hline
        Problem-3 & 372 & 52184 & 16681 & 589 \\
                                   
        \hline
        Problem-4 & 354 & 9264 & 2836 & 92 \\
                                   
        \hline
    \end{tabular}
    \caption{Comparison of performance metrics for different problems.}
    \label{tab:results}
\end{table}

The table evaluates the following several key performance metrics for each approach:

\begin{itemize}
    \item \textbf{Plan Cost:} Represents the cumulative cost of executing all actions in a plan. Lower costs indicate more efficient solutions.
    
    \item \textbf{Nodes Generated:} Provides insight into the computational effort of the search algorithm by indicating the total number of states considered.
    
    \item \textbf{Nodes Expanded:} Shows how many of the generated states were fully explored, offering a measure of the search's thoroughness.
    
    \item \textbf{Total Time (ms):} Reflects the duration taken by the algorithm to find a solution, in miliseconds. Faster times are advantageous, especially in time-sensitive scenarios.
\end{itemize}

Our results provide a comparative analysis across different problems. The Plan Cost consistently increases, as expected, due to the growing complexity of the problems. The Nodes Generated and Expanded, as well as the Total Time, show a high correlation with the Plan Cost, increasing when the latter does. This is a logical result, since the more steps a plan requires, the more exhaustive the search will be in the Plan Search Space. There is, however, a noticeable outlier in the metrics, with Problem 3. The fact that it has slightly more steps than Problem 4 (which we can remember is the same scenario with different priorities) indicates that one of the two orderings of dishes causes the robot to act more efficiently. This can be due to some dishes involving more complexity than others, and said complexity being more (or less) mitigated when found in specific parts of the plan. One more thing to be noticed about Problem 3 is that, despite having a slightly higher Plan Cost than Problem 4, the Nodes Generated and Expanded, as well as the Total Time, seem to spike out of proportion, which shows even more clearly that the solution to Problem 3 is much harder to find when searching the Plan Space, despite seeming very similar to Problem 4 from our perspective. 

\newpage
\section{Conclusion}

This project successfully demonstrated the development of a sophisticated PDDL-based planning model for a robot chef operating in a complex kitchen environment. Through a systematic approach of defining predicates, actions, and constraints, we created a flexible and adaptable planning system capable of handling multiple cooking and serving scenarios.

\subsection{Key Achievements}

\begin{enumerate}
    \item \textbf{Comprehensive Kitchen Modeling}: We developed a robust domain that captures the intricate dynamics of a kitchen, including:
    \begin{itemize}
        \item Detailed room-based layout
        \item Complex ingredient preparation and cooking processes
        \item Tool management and cleaning
        \item Order prioritization
    \end{itemize}

    \item \textbf{Flexible Action Representation}: The model's actions, such as \texttt{move}, \texttt{pick-up}, \texttt{prepare-ingredient}, and \texttt{serve-dish}, provide a comprehensive framework for robotic kitchen operations.

    \item \textbf{Scenario Testing}: Our four test cases progressively validated the model's capabilities, demonstrating its ability to:
    \begin{itemize}
        \item Navigate kitchen spaces
        \item Handle multiple dishes
        \item Manage ingredient quantities
        \item Respect dish priority orders
    \end{itemize}
\end{enumerate}

\subsection{Performance Insights}

The performance analysis revealed critical computational characteristics of the planning model:
\begin{itemize}
    \item Plan complexity directly correlates with computational effort
    \item Dish priority and order can significantly impact search space efficiency
    \item The model becomes more computationally intensive with increased task complexity
\end{itemize}

\subsection{Future Directions}

While the current implementation provides a solid foundation, potential improvements could include:
\begin{itemize}
    \item More sophisticated ingredient quantity management
    \item Enhanced priority handling mechanisms
    \item Integration with real-world robotic systems
\end{itemize}

The successful implementation of this PDDL model represents a significant step towards automated, intelligent kitchen assistance, showcasing the potential of AI planning techniques in complex, real-world scenarios.


% In this report, we explored two distinct approaches for modeling a rescue drone tasked with saving individuals in an emergency grid environment: the "Adjacency List" and the "Coordinate System" methods. Through a series of test cases, we evaluated the performance of each approach in terms of Plan Cost, Nodes Generated, Nodes Expanded, and Total Time.
% 
% \vspace{1em}
% 
% Our analysis revealed that while both approaches achieve optimal Plan Costs, the "Adjacency List" method consistently outperforms the "Coordinate System" in terms of computational efficiency, particularly as the complexity of the problems increases. This is evidenced by the lower number of nodes generated and expanded, as well as the reduced total time required to find solutions. The "Coordinate System" approach, although more concise in its problem definition, tends to explore a larger search space, leading to increased computational effort and time.
% 
% \vspace{1em}
% 
% Interestingly, in simpler scenarios such as Problems 1 and 2, both methods required minimal exploration, resulting in zero nodes generated and expanded. However, as the complexity increased, the "Adjacency List" approach demonstrated superior efficiency, making it a more suitable choice for complex problem-solving scenarios.
% 
% \vspace{1em}
% 
% This study emphasizes the importance of selecting the right modeling approach based on the specific needs and constraints of the problem to achieve the best results effectively and efficiently. Additionally, exploring the use of fluents or comparing our methods with other planners could provide further insights and improvements.

\end{document}